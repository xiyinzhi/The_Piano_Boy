\documentclass[conference]{IEEEtran}
\IEEEoverridecommandlockouts
% The preceding line is only needed to identify funding in the first footnote. If that is unneeded, please comment it out.
\usepackage{cite}
\usepackage{amsmath,amssymb,amsfonts}
\usepackage{algorithmic}
\usepackage{graphicx}
\usepackage{textcomp}
\usepackage{xcolor}
\usepackage{url}
\def\BibTeX{{\rm B\kern-.05em{\sc i\kern-.025em b}\kern-.08em
    T\kern-.1667em\lower.7ex\hbox{E}\kern-.125emX}}
\begin{document}

\title{The Piano Boy: An Optimized Music Generator\\
{\footnotesize \textsuperscript{}}
\thanks{}
}

\author{\IEEEauthorblockN{Wei Liu}
\IEEEauthorblockA{\textit{Department of Computer Science} \\
\textit{Georgetown University}\\
Washington, D.C., USA \\
wl521@georgetown.edu}
\and
\IEEEauthorblockN{Yinzhi Xi}
\IEEEauthorblockA{\textit{Department of Computer Science} \\
\textit{Georgetown University}\\
Washington, D.C., USA \\
yx157@georgetown.edu}
\and
\IEEEauthorblockN{Jie He}
\IEEEauthorblockA{\textit{Department of Analytics} \\
\textit{Georgetown University}\\
Washington, D.C., USA \\
jh2063@georgetown.edu}
}

\maketitle

\begin{abstract}
Being curious about the creativity of neural networks, we developed a neural network to generate music. In this paper, a 3-layer LSTM model is used as the basis for music development. Several improvements have been applied to increase the performance of our model successfully. We also have made some modifications to the model structure as comparisons and tested them. The final results show that our optimized model performs best among our comparative experiments and some related work.
\end{abstract}

\begin{IEEEkeywords}
note, LSTM, matrix, binary cross entropy, sigmoid
\end{IEEEkeywords}

\section{Introduction}
Artificial intelligence is playing a much bigger role than ever. Recently, deep learning is the hottest and most exciting topic in this area. Neural networks are widely used to improve many aspects of our daily life. People used them to improve product recommendation systems, make better predictions about sports scores, generate artwork with transferred style, and so on. Since we all love music, we want to explore how neural networks can help with creating good music. It will be nice for a music fan to have some fun with a decent music generator. The goal/motivation is to improve existing AI music composers by using different neural network architectures or adding creative elements to the models. Is it possible to create an AI music composer which can produce realistic music pieces like those written by human beings? Is it possible to overcome the problem of lack of rhythm which is shared by most music generators on the market? We need solutions for them. 
 
We chose a model with 3 LSTM layers as the foundation and made big improvements from there. One of them is to make our output music possess changing rhythm and we added some pauses to the training samples to achieve that. The other one is that we changed the inputs from single note and chord to a matrix of 88 notes (each represented by 0 or 1) to make our outputs more flexible. Besides, we also made some modifications to the model structure, tested these modified models, and used them as comparisons. Since we were building a generative model, evaluation can rely on the feedbacks of listeners. The final evaluation results show that our optimized model was the best performer among our comparative experiments and some similar applications.   


\section{Related Work}

We conducted some research about previous and related methods which have been used to solve similar problems. McMahon, B.[1] built AI Jukebox by using a Bidirectional LSTM architecture in Keras. Kim, J.[4] created deepjazz, which was built on a model with 2 LSTM layers, in Keras and Theano. Svegliato, J. brought us Deep Jammer which was also built on a model with 2 LSTM layers in Theano. Sigurgeirsson, S.[2] presented Classical Piano Composer which employed a DNN model with 3 LSTM layers. Based on our research, LSTM is commonly used in solving similar music generating problems. LSTM is considered as an extremely useful solution to deal with problems in which a network must remember information for a long period of time like music generation. LSTM can effectively represent both the long-term and short-term sequential dependencies in music. The Stacked LSTM has multiple hidden LSTM layers where each layer contains multiple memory cells and it is an extension to the original LSTM model. We picked a 3-layer stacked LSTM model as the starting point and foundation for our project. 

\section{Datasets}
The dataset used in the project is a combination of collections of midi files, which are all classical piano music:
\begin{itemize}
\item midi\_songs: 92 pieces
\item chopin: 98 pieces
\end{itemize}

A midi file is a Musical Instrument Digital Interface file and it explains what notes are played and how long or loud each note should be. For data pre-processing, we used \textit{Music21}[6], a Python toolkit developed by MIT, to extract musical notation of midi files like notes and chords.

For each dataset of above two datasets, we split midi pieces into training set and test set. Each piece of midi files can generate several sequences for corresponding inputs and outputs, with sequence length of 100. Our model uses previous 100 sequence as the input and try to predict the 101st sequence. For midi\_songs dataset, we have 46310 items in training set, for chopon dataset, we have 58286 items in training set.

Each time we want to generate a piece of music, our program choose a sequence of 100 from the test set randomly as the input and output the generated music.

\section{Methods}

\subsection{Transforming MIDI files}
We use \textit{Music21} library, which is a open-source library for Python, to extract note and chord information from midi files.

\subsection{Basic Structure}
We use a 5 layers neural network to do this music generation task, it includes 3 LSTM layers, each LSTM layer contains 512 nodes, 1 full-connection layer with size 256 and 1 output layer. The input of the neural network is 100 beat pieces, the piece is encoded into a vector, the vector length varies in our different implementations. The output of the neural network is one beat piece. In the training stage, we generate training set from our midi files and feed it into our 5 layers model, in the generating stage, we randomly select a 100 beat piece as model input, model predicts the next beat piece and append it into input, remove the first beat piece and repeat this procedure several times.

\subsection{Functions}
In the initial implementation, we treat each metre piece as a word, basically, each metre piece is either a note word or a chord word, then we use one hot encoding to transform each metre to a vector, the vector dimension is the size of unique note and chord. The output dimension is also the size of unique note and chord. Since every time we need to predict a note or chord, we use \textit{softmax} as the activation function of the output layer and use \textit{categorical cross entropy} as the loss function.

\subsection{Incorporating Rest Note}
The initial implementation cannot output a rest note, our next implementation is to incorporate rest note, to achieve this, we read the midi file every 0.5 seconds, if there is no note or chord at that moment, we add an empty note. So the input and output vector dimension increases by one.

\subsection{Replacing Note/Chord to Vector}
Our next improvement is focused on note/chord representation, in the last two implementations, the note/chord is represented as a word (String), instead, we use an 88 length vector to represent them since piano has 88 keys. We still use \textit{softmax} as the activation function so our model can only predict the next note and does not support multiple keys pressed at the same time.

\subsection{Multiple Keys Prediction}
The final improvement is to support multiple keys prediction. In order to generate multiple 1s in the output vector, we modify the activation function from \textit{softmax} to \textit{sigmoid} and use a threshold 0.5 to determine if a key is predicted as pressed or not. To work with \textit{sigmoid} together, we use \textit{binary cross entropy} as the loss function.

\subsection{Regularization}
We use dropout and early-stop to avoid overfitting, the dropout rate is 0.3.

\section{Results}
\subsection{Models without 88 keys Representation}
The Stacked LSTM has multiple hidden LSTM layers where each layer contains multiple memory cells and it is an extension to the basic LSTM model. We started from a 3-layer stacked LSTM model and made some modifications to it. We employed the following variations: 
\begin{itemize}
\item Four LSTM Layers: Add one additional hidden LSTM layer as the fourth LSTM layer.
Dimensions: (512,512,512,512)
\item Three LSTM Layers. 
Dimensions: (512,512,512)
\item Three LSTM Layers while model input and output contains rest notes.
Dimensions: (512,512,512)
\end{itemize}

See training loss of different LSTM models in Fig.~\ref{fig1}.

\begin{figure}[htbp]
\centerline{\includegraphics[scale=0.5]{pics/compare1.png}}
\caption{Training Loss of Different LSTM Models}
\label{fig1}
\end{figure}

Although the 4-layer LSTM had the best training result, but after we evaluated models by generating several midi files, we found 4-layer LSTM model had a very bad performance on generating music.

It takes 20 hours to run 200 epochs with a NVIDIA GTX 1080 GPU to train each of above 3 models.

\subsection{Models with 88 keys Representation}
We employed the follwing two different models with 88 keys representation:
\begin{itemize}
\item 88 keys model 1, it predicts multiple possible keys at one beat, it use sigmoid as activation function of output layer and binary\_crossentropy as loss function, we also use a threshold 0.5 to determine if a key is pressed or not.
\item 88 keys model 2, it only predict the key with the most probability, it use softmax as activation function of output layer and categorical\_crossentropy as loss function.
\end{itemize}

See training loss of above models in Fig.~\ref{fig2}.
\begin{figure}[htbp]
\centerline{\includegraphics[scale=0.5]{pics/compare2.png}}
\caption{Training Loss of 88 keys LSTM Models}
\label{fig2}
\end{figure}

See training accuracy of above models in Fig.~\ref{fig3}.
\begin{figure}[htbp]
\centerline{\includegraphics[scale=0.5]{pics/compare3.png}}.
\caption{Training Accuracy of 88 keys LSTM Models}
\label{fig3}
\end{figure}

We can see the initial model with rest note became overfitting at 125 epoch.

After about 190 epochs, the 88 keys model 1 reached the best performance on training dataset.
In order to compare the training effects with different epochs, we generated several midi files with 1, 50, 100, 152, 191 epochs separately. See Key representation of generated midi files in Fig.~\ref{fig4}.

\begin{figure}[htbp]
\centerline{\includegraphics[scale=0.45]{compare_results.png}}
\caption{Key representation of generated midi files in different epochs\protect\footnotemark}
\label{fig4}
\end{figure}
\footnotetext{from top to bottom are test samples with 1, 50, 100, 152, 191 epochs separately}

We can see that at the beginning, the model cannot generate midi at all, and at 100 epoch, it only produced the same beat repeatly. While the training continued, the model performance became better and better.

See validation accuracy of 88 keys model 1 in Fig.~\ref{fig5}.
\begin{figure}[htbp]
\centerline{\includegraphics[scale=0.5]{pics/compare5.png}}.
\caption{Accuracy of 88 keys LSTM Models}
\label{fig5}
\end{figure}

We can see that on validation dataset, the accuracy of 88 keys model 1 became worse while training was continue, the reason of the high accuracy value at the beginning is that we have many 0s in the 88 keys representation, since at one beat, there are at most 5-10 keys pressed, so the baseline of accuracy is around 90\%.

We believe that the same 100 previous beat pieces cannot guarantee that the following keys are unique, so we don't use accuracy as the measure. Instead, we let people to listen to our generated music and music from some related work and vote for the best music. See results in Table.~\ref{tab1}.

\begin{table}[htbp]
\caption{Voting Results}
\begin{center}
\begin{tabular}{|c|c|}
\hline
Program Name& Number of Votes\\
\hline
AI Jukebox& 7\\
Classical Piano Composer& 7\\
Deep jazz& 5\\
Deep Jammer& 4\\
\textbf{The Piano Boy}& 18\\
\hline
\end{tabular}
\label{tab1}
\end{center}
\end{table}

\section{Discussion of Results}
Most work for generating music use LSTM as the models,  and we also use a 3-layer LSTM as the basis of our model structure. We have tried some models as comparison, but it seems their performances are not as good as our optimized model. For the 4-layer LSTM model, its loss is low but its generated music contains few notes and always repeating several notes. As comparison, the 3-layer LSTM model works better based on our results.

Compared with other models, some of them can only generate one note at a time(e.g. Deep Jazz), our model can produce several notes at a time, making the music more rich and sound, which is an great improvement in performance. Our final model can predict rest note which output is all 0s and also can predict any combination of note and chord which output can have multiple 1s, which make the rhythm of the melody flexible. Another common problem for generated music by related work is, many will end up repeating of one note. Our model greatly alleviates the problem after a series of improvements. The music generated by our model usually has changeable melodies. In generated music, we can see sometimes it looks like our model can memorize the training music, which is really amazing. 

\section{Conclusions}
To improve the quality of generated music, we tried several different model structures, such as a three-LSTM-layer model with different dimensionality and a four-LSTM-layer model.

We also tried to use different beat lengths as input like using 40 instead of 100. However, the resulting outputs were not good.

Our final model is simply superior to the other experiments mentioned above. Of course, it is also better than some other similar applications chosen as our project baselines based on the survey result.

Now our model's prediction result is fixed if the input is determined, but we believe that music composition should be vibrant which means that same input could lead to different output. So we think that a probabilistic neural network model could be used in future work.

\begin{thebibliography}{00}
\bibitem{b1} McMahon, B. \textit{Music generator utilizing a Bidirectional LSTM architecture in Keras,} \url{https://github.com/cipher813/AI_Jukebox}.
\bibitem{b2} Sigurgeirsson, S. \textit{Classical Piano Composer,} \url{https://github.com/Skuldur/Classical-Piano-Composer}.
\bibitem{b3} Justin Svegliato and Sam Witty, \textit{Deep Jammer: A Music Generation Model}.
\bibitem{b4} Kim, J. \textit{Deep learning driven jazz generation using Keras \& Theano!,} \url{https://github.com/jisungk/deepjazz}.
\bibitem{b5} Ian Goodfellow, Yoshua Bengio, Aaron Courville. Deep Learning[M]. MIT Press, 2016.
\bibitem{b6} M.S. Cuthbert and C. Ariza. music21: A toolkit for computer-aided musicology and symbolic music data. In \textit{9th International Society for Music Information Retrieval Conference}, pages 637-642, Utrecht, Netherlands, August 2010.
\bibitem{b7} Hochreiter, S. and Schmidhuber, J. Long short-term memory. \textit{Neural Comput}. 9, 1735–1780 (1997).

\end{thebibliography}
\vspace{12pt}


\end{document}
